% !Mode:: "TeX:UTF-8"

%% LaTeX article base template for two-column layout with title page and header etc
% Mark Verleg, 2014

\documentclass[a4paper,twoside,UTF8]{article}

%% LOAD PACKAGES

%% Load packages required for any half-functional setup.


%% LOAD PACKAGES
\usepackage{nag}		% complain about any problems found
\usepackage{graphicx}	% for including images using \begin{figure}[ht] \includegraphics[keepaspectratio]{name.png}
\usepackage{mathtools}	% math formulas, with $ equation $ or \begin{equation}, apparently better than amsmath
\usepackage[colorlinks=true,urlcolor=blue,linkcolor=black,citecolor=black]{hyperref}	% allows using simple URLs without infinite escape chars
\usepackage{etoolbox}	% allows toggles and if statements

%% CONFIGURE PACKAGES
\mathtoolsset{showonlyrefs}




%% Load packages useful for some, especially larger, projects.


%% LOAD PACKAGES
\usepackage{multicol}	% to have selective areas, marged with \begin{multicols}{2}, have multiple columns
\usepackage{fancyhdr}	% allows adding headers and/or footers to the document
\usepackage{adjustbox}	% scale things (pgf images) the same way as normal images
\usepackage{lastpage}	% get a reference to the last page (for page numbering)
\usepackage{xcolor}	% provides \textcolor
\usepackage{layouts}	% just for outputting linewidths etc
\usepackage{tikz}		% drawing library, related to pgf
\usepackage{pgf}		% to show .pgf images (as generated by matplotlib)
\usepackage[backend=bibtex]{biblatex} % cite bibtex references
\usepackage{caption}	% align figure captions
\usepackage{braket}		% <a|b> notation, \Bra{}, \Braket{|}
\usepackage[gobble=auto]{pythontex}	% style and run python code (order matters for some reason)
\usepackage{fontspec}	% for setting fonts (for e.g. Chinese text)
\usepackage{listings}   % source code display with \begin{lstlisting}

%% CONFIGURE PACKAGES
\captionsetup[figure]{slc=off}	% align figure captions left
\newtoggle{electronic}
\newtoggle{finalversion}
\newtoggle{headers}
\toggletrue{electronic}
\toggletrue{headers}
\toggletrue{finalversion}




%% DEFINITIONS
% properties
\toggletrue{finalversion}
\renewcommand{\authorname}{Mark Verleg}
\renewcommand{\departmentname}{DICP}
\renewcommand{\departmentlogo}{radboud.png}
\renewcommand{\shorttitle}{Interesting document}
\renewcommand{\longtitle}{Amazingly interesting and fascinating document with annoyingly lengthy name}

%% CUSTOM COMMANDS

% Load several custom commands.


%% CUSTOM COMMANDS
\newcommand{\comment}[1]{}
\newcommand{\todo}[1]{\textbf{ \#TODO }}
\newcommand{\latin}[1]{{\it #1}}
\newcommand{\pgfimg}[1]{\includegraphics{#1.pdf}}
\newcommand{\topwidemarker}[0]{\par\noindent\rule{\dimexpr(0.5\textwidth-0.5\columnsep-0.4pt)}{0.4pt}\rule{0.4pt}{6pt}}
\newcommand{\bottomwidemarker}[0]{\vspace{\belowdisplayskip}\hfill\rule[-6pt]{0.4pt}{6.4pt}\rule{\dimexpr(0.5\textwidth-0.5\columnsep-1pt)}{0.4pt}}
\newcommand{\cn}[1]{{\setmainfont{WenQuanYi Zen Hei} #1 }} % find Chinese fonts on Ubuntu: fc-list :lang=zh (make sure they are valid; name not specified in more than one language)
\newcommand{\exercise}[1]{\begin{quote} \emph{ #1 } \end{quote}}




%% REFERENCES
\addbibresource{library.bib}

%% LAYOUT SETTINGS

%% TeX settings for article layout.


%% PAGE LAYOUT
% margin different for paper and electronic form
\iftoggle{electronic}{%
	\setboolean{@twoside}{false}
}{}

% two-column layout (multicol package) because long lines are apparently bad, but default margins are crazy
% use \begin{multicols}{2} ... \end{multicols} and place images outside that
\addtolength{\oddsidemargin}{-0.7in}
\addtolength{\evensidemargin}{-0.7in}
\addtolength{\textwidth}{1.4in}
\setlength{\columnsep}{0.5in}

%% HEADER & FOOTER
\iftoggle{headers}{%
	% hope '\leftmark' is always the short section title, because I can't find a direct reference...
	\pagestyle{fancy}
	\renewcommand{\headrulewidth}{0.2pt}
	\renewcommand{\footrulewidth}{0.0pt}
	\fancyhead[C]{page \thepage\ of \pageref{LastPage}}
	\fancyfoot[L,C]{}
	% some header and footer even/odd changes for printed format
	\iftoggle{electronic}{%
		\fancyhead[LE]{\authorname , \departmentname}
		\fancyhead[LO]{\MakeUppercase{\shorttitle}}
		\fancyhead[R]{\leftmark}
		\ifdefempty{\departmentlogo}{}{
			\fancyfoot[R]{\raisebox{-1.0\height}{\includegraphics[height=0.8in,keepaspectratio]{\departmentlogo}}}
		}
	}{%
		\fancyhead[RE]{\authorname , \departmentname}
		\fancyhead[LO]{\MakeUppercase{\shorttitle}}
		\fancyhead[LE,RO]{\leftmark}
		\ifdefempty{\departmentlogo}{}{
			\fancyfoot[RO]{\raisebox{-1.0\height}{\includegraphics[height=0.8in,keepaspectratio]{\departmentlogo}}}
		}
	}
	% add 'concept version' footer if this is, in fact, a concept version
	\iftoggle{finalversion}{}{%
		\fancyfoot[L]{\textcolor{red}{\\ \MakeUppercase{Concept version}} \\ \today }
	}
}{}

%% DEFAULT VALUES
\newcommand{\authorname}{Mark Verleg \textbackslash authorname}
\newcommand{\departmentname}{Department \textbackslash departmentname}
\newcommand{\departmentlogo}{}
\newcommand{\shorttitle}{Short title \textbackslash shorttitle}
\newcommand{\longtitle}{Long title \textbackslash longtitle}




%% DOCUMENT
\begin{document}

	%% TITLE PAGE
	% header page content
	\title{\longtitle}
	\author{\authorname \\ \departmentname}
	\date{\today}
	\maketitle

	% Abstract
	\begin{abstract}

	\end{abstract}

	% Concept version?
	\iftoggle{finalversion}{}{%
		\begin{center}
			~\\ ~\\ ~\\ ~\\ ~\\ ~\\
			\rotatebox{22}{\textcolor{red}{\huge{CONCEPT VERSION}}}
		\end{center}
	}

	\newpage

	%% TABLE OF CONTENT PAGE(S)
	\tableofcontents
	\newpage

	%% CONTENT
	\section[Short Section Title]{Long Section Title}
		\begin{multicols}{2}
		\subsection{}
			\paragraph{}
				Lorem ipsum dolor sit amet, consectetur adipiscing elit. Nam tempor gravida justo ac egestas. Sed consequat vehicula adipiscing. Vestibulum ante ipsum primis in faucibus orci luctus et ultrices posuere cubilia Curae; Sed iaculis, nisi non consequat placerat, lectus magna fringilla leo, in elementum leo elit eget eros. Pellentesque sagittis nisl eu rutrum fermentum. Quisque quis arcu vulputate, rhoncus odio ac, vulputate purus. Aliquam posuere urna vitae gravida vulputate. Duis justo est, pretium nec tellus id, fringilla mattis purus. Sed et imperdiet leo. Aliquam auctor condimentum orci. Phasellus viverra turpis dui, commodo scelerisque massa rutrum vulputate. Suspendisse pellentesque, nisi et imperdiet suscipit, tellus lectus interdum nunc, ut cursus sapien enim vitae felis. Sed ultricies rhoncus nisl, et hendrerit justo semper ut.
			\paragraph{}
				font: \fontname\font\ \the\fontdimen6\font
				line: \printinunitsof{in}\prntlen{\textwidth} \printinunitsof{in}\prntlen{\linewidth}
			\paragraph{}
				Does this show (four) simplified Chinese characters? 努力工作 \cny{努力工作}. Check if the font is installed if it doesn't.
			\paragraph{}
				Do you see an equation and two images?
		\end{multicols}

		\begin{equation} \begin{split}
			\frac{\sum_{i=1}^{n-1}N_{i}N_{i+1}}{\sum_{i=1}^{n} N_{i}}
			\label{eq:equation_label}
		\end{split} \end{equation}

		\begin{figure}[ht]
			\includegraphics[width=\linewidth,keepaspectratio]{demo.png}
			\caption{PNG; comparison of figure width and font size}
			\label{fig:png_label}
		\end{figure}

		\begin{figure}[ht]
			\input{demo.pgf}
			\caption{PGF; comparison of figure width and font size}
			\label{fig:pgf_label}
		\end{figure}

\end{document}


%% IMAGES
% images generated by matplotlib can be given the correct dimensions
% and fonts automatically by using MyMPL to generate them; to find text
% width and font settings, render in main body (with layouts package):
%	font: \fontname\font\ \the\fontdimen6\font
%	line: \printinunitsof{in}\prntlen{\textwidth} \printinunitsof{in}\prntlen{\linewidth}
% PGF images are by default rendered to seperate PDF files first, which
% are then included (faster & less memory problems); see makefile
% to include an image which can be referenced by \ref{fig:figure_label} use either:
%	\begin{figure}[ht]
%		\pgfimg{name_without_extension}
%		\caption{enter a caption here}
%		\label{fig:figure_label}
%	\end{figure}
% for PGF; for PNG use (possibly without [...]):
%		\includegraphics[width=\linewidth,keepaspectratio]{demo.png}

%% EQUATIONS
% small formulas can be placed between $ and $ , numbered equations
% are made like this:
%	\begin{equation} \begin{split}
%		\frac{\sum_{i=1}^{n-1}N_{i}N_{i+1}}{\sum_{i=1}^{n} N_{i}}
%		\label{eq:equation_label}
%	\end{split} \end{equation}
% where split let's you split the equation with \\ (no automatic way)

%% TABLES
% a simple table can be made like this:
%	\begin{table}
%		\begin{tabular}{| l | l | r | r |} \hline
%			underlined left cell & same & underlined right cell & same \\ \hline
%			left cell & left cell & right cell & right cell \\ \hline
%		\end{tabular}
%		\caption{enter a caption here}
%		\label{tab:table_label}
%	\end{table}

%% LISTS
%	\begin{enumerate} % or 'itemize'
%		\item
%		\item
%	\end{enumerate}


