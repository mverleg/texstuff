% !Mode:: "TeX:UTF-8"

\documentclass[a4paper,UTF8]{article}

%% LOAD PACKAGES
\usepackage{graphicx}	% for including images using \begin{figure}[ht] \includegraphics[keepaspectratio]{name.png}
\usepackage{mathtools}	% math formulas, with $ equation $ or \begin{equation}, apparently better than amsmath
\usepackage[colorlinks=true,urlcolor=blue,linkcolor=black,citecolor=black]{hyperref}	% allows using simple URLs without infinite escape chars
\usepackage{fancyhdr}	% allows adding headers and/or footers to the document
\usepackage{caption}	% align figure captions
% \usepackage{pgf}		% to show .pgf images (as generated by matplotlib)
% \usepackage{tikz}		% drawing library, abstraction above pgf
\usepackage{gnuplottex} \usepackage{epstopdf}
% \usepackage{adjustbox}	% scale things (pgf images) the same way as normal images
% \usepackage{lastpage}	% get a reference to the last page (for page numbering)
% \usepackage{xcolor}	% provides \textcolor
\usepackage{layouts}	% just for outputting linewidths etc
% \usepackage{braket}		% <a|b> notation, \Bra{}, \Braket{|}
\usepackage[quiet]{fontspec}	% for setting fonts (for e.g. Chinese text)
% \usepackage{listings}   % source code display with \begin{lstlisting}
\usepackage[backend=bibtex]{biblatex} % cite bibtex references
\usepackage[gobble=auto]{pythontex}	% style and run python code (order matters for some reason)

%% LAYOUT SETTINGS
% mist settings
\mathtoolsset{showonlyrefs}     % only number equations which are referenced
\captionsetup[figure]{slc=off}	% align figure captions left

% wider layout (and two-column settings)
\addtolength{\oddsidemargin}{-0.7in}
\addtolength{\evensidemargin}{-0.7in}
\addtolength{\textwidth}{1.4in}
\setlength{\columnsep}{0.5in}

%% HEADER & FOOTER
% hope '\leftmark' is always the short section title, because I can't find a direct reference...
\pagestyle{fancy}
\renewcommand{\headrulewidth}{0.2pt}
\renewcommand{\footrulewidth}{0.0pt}
\fancyhead[C]{page \thepage\ of \pageref{LastPage}}
\fancyfoot[L,C]{}
% header and footer for electronic format
\fancyhead[L]{\authorname , \departmentname}
\fancyhead[L]{\MakeUppercase{\shorttitle}}
\fancyhead[R]{\leftmark}
\fancyfoot[R]{\raisebox{-1.0\height}{\includegraphics[height=0.8in,keepaspectratio]{\departmentlogo}}}
% for printed format, uncomment this:
%\fancyhead[RE]{\authorname , \departmentname}
%\fancyhead[LO]{\MakeUppercase{\shorttitle}}
%\fancyhead[LE,RO]{\leftmark}
%\fancyfoot[RO]{\raisebox{-1.0\height}{\includegraphics[height=0.8in,keepaspectratio]{\departmentlogo}}}
%% ALSO ADD twoside TO DOCUMENTCLASS

% PROPERTIES
\newcommand{\authorname}{Mark Verleg}
\newcommand{\departmentname}{DICP}
\newcommand{\departmentlogo}{radboud.png}
\newcommand{\shorttitle}{Interesting document}
\newcommand{\longtitle}{Amazingly interesting and fascinating document with annoyingly lengthy name}

% concept version?
\fancyfoot[L]{\textcolor{red}{\\ \MakeUppercase{Concept version}} \\ \today }

%% CUSTOM COMMANDS
%\newcommand{\comment}[1]{}
\newcommand{\todo}[1]{\textbf{ \#TODO }}
\newcommand{\latin}[1]{{\it #1}}
\newcommand{\cn}[1]{{\setmainfont{WenQuanYi Zen Hei} #1 }} % find Chinese fonts on Ubuntu: fc-list :lang=zh (make sure they are valid; name not specified in more than one language)
\newcommand{\degree}[0]{^{\circ}}
% \newcommand{\pgfimg}[1]{\includegraphics{#1.pdf}}

%% REFERENCES
\addbibresource{library.bib}


%% DOCUMENT
\begin{document}

	%% TITLE PAGE
	% header page content
	\title{\longtitle}
	\author{\authorname \\ \departmentname}
	\date{\today}
	\maketitle

	% Abstract
	\begin{abstract}

	\end{abstract}

	% concept version?
	\begin{center}
		~\\ ~\\ ~\\ ~\\ ~\\ ~\\
		\rotatebox{22}{\textcolor{red}{\huge{CONCEPT VERSION}}}
	\end{center}

	\newpage

	%% TABLE OF CONTENT PAGE(S)
	\tableofcontents
	\newpage

	%% CONTENT
	\section[Short Section Title]{Long Section Title}
		\subsection{}
			\paragraph{}
				Lorem ipsum dolor sit amet, consectetur adipiscing elit. Nam tempor gravida justo ac egestas. Sed consequat vehicula adipiscing. Vestibulum ante ipsum primis in faucibus orci luctus et ultrices posuere cubilia Curae; Sed iaculis, nisi non consequat placerat, lectus magna fringilla leo, in elementum leo elit eget eros. Pellentesque sagittis nisl eu rutrum fermentum. Quisque quis arcu vulputate, rhoncus odio ac, vulputate purus. Aliquam posuere urna vitae gravida vulputate. Duis justo est, pretium nec tellus id, fringilla mattis purus. Sed et imperdiet leo. Aliquam auctor condimentum orci. Phasellus viverra turpis dui, commodo scelerisque massa rutrum vulputate. Suspendisse pellentesque, nisi et imperdiet suscipit, tellus lectus interdum nunc, ut cursus sapien enim vitae felis. Sed ultricies rhoncus nisl, et hendrerit justo semper ut.
			\paragraph{}
				font: \fontname\font\ \the\fontdimen6\font
				line: \printinunitsof{in}\prntlen{\textwidth} \printinunitsof{in}\prntlen{\linewidth}
			\paragraph{}
				Does this show (four) simplified Chinese characters? \cn{努力工作}. Check if the font is installed if it doesn't.
			\paragraph{}
				Do you see an equation and two images?

		\begin{align}
			\frac{\sum_{i=1}^{n-1}N_{i}N_{i+1}}{\sum_{i=1}^{n} N_{i}}
			\label{eq:equation_label}
		\end{align}

		%\begin{figure}[ht]
		%	\includegraphics[width=\linewidth,keepaspectratio]{demo.png}
		%	\caption{PNG; comparison of figure width and font size}
		%	\label{fig:png_label}
		%\end{figure}

		%\begin{figure}[ht]
		%	\pgfimg{demo}
		%	\caption{PGF; comparison of figure width and font size}
		%	\label{fig:pgf_label}
		%\end{figure}
		
		\begin{figure}[htbp]
			\centering
			\includegraphics{genimg\string~/default.pdf}
			\caption{PS gnuplot image}
			\label{yolo}
		\end{figure}
		
		\subsection{}
			\paragraph{}
				This is what a link looks like: \href{http://tex.stackexchange.com/questions/120154/is-latex-nowadays-still-that-superior-to-word}{word vs latex}
			\paragraph{}
				This should be a simple Python command promp:
			\paragraph{}
				\py{2+2}


\end{document}


