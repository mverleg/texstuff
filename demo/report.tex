% !Mode:: "TeX:UTF-8"

%% LaTeX report base template for simple layout
% Mark Verleg, 2014

\documentclass[a4paper,UTF8]{article}

%% TOP HEADER
\include{minimal}
% \include{extra}

%% CUSTOM COMMANDS

% Load several custom commands.


%% CUSTOM COMMANDS
\newcommand{\comment}[1]{}
\newcommand{\todo}[1]{\textbf{ \#TODO }}
\newcommand{\latin}[1]{{\it #1}}
\newcommand{\pgfimg}[1]{\includegraphics{#1.pdf}}
\newcommand{\topwidemarker}[0]{\par\noindent\rule{\dimexpr(0.5\textwidth-0.5\columnsep-0.4pt)}{0.4pt}\rule{0.4pt}{6pt}}
\newcommand{\bottomwidemarker}[0]{\vspace{\belowdisplayskip}\hfill\rule[-6pt]{0.4pt}{6.4pt}\rule{\dimexpr(0.5\textwidth-0.5\columnsep-1pt)}{0.4pt}}
\newcommand{\cn}[1]{{\setmainfont{WenQuanYi Zen Hei} #1 }} % find Chinese fonts on Ubuntu: fc-list :lang=zh (make sure they are valid; name not specified in more than one language)
\newcommand{\exercise}[1]{\begin{quote} \emph{ #1 } \end{quote}}




%% LAYOUT SETTINGS

%% TeX settings for article layout.


%% HEADER & FOOTER
\pagenumbering{arabic}




%% DOCUMENT
\begin{document}

	%% TITLE PAGE
	% header page content
	\title{\longtitle}
	\author{\authorname \\ \departmentname}
	\date{\today}
	\maketitle

	% Abstract
	\begin{abstract}

	\end{abstract}

	% concept version?
	\iftoggle{finalversion}{}{%
		\begin{center}
			~\\ ~\\ ~\\ ~\\ ~\\ ~\\
			\rotatebox{22}{\textcolor{red}{\huge{CONCEPT VERSION}}}
		\end{center}
	}

	\newpage

	%% TABLE OF CONTENT PAGE(S)
	\tableofcontents
	\newpage

	%% CONTENT
	\section{Long Section Title}
		
		\paragraph{}
		
			\exercise{Lorum ipsum?}
		
		\paragraph{}
			font: \fontname\font\ \the\fontdimen6\font
			\\
			line: \printinunitsof{in}\prntlen{\textwidth} \printinunitsof{in}\prntlen{\linewidth}

		\begin{equation} \begin{split}
			\frac{\sum_{i=1}^{n-1}N_{i}N_{i+1}}{\sum_{i=1}^{n} N_{i}}
			\label{eq:equation_label}
		\end{split} \end{equation}

		\begin{figure}[ht]
			\includegraphics[width=\linewidth,keepaspectratio]{demo.png}
			\caption{PNG; comparison of figure width and font size}
			\label{fig:png_label}
		\end{figure}

\end{document}


%% IMAGES
% images generated by matplotlib can be given the correct dimensions
% and fonts automatically by using MyMPL to generate them; to find text
% width and font settings, render in main body (with layouts package):
%	font: \fontname\font\ \the\fontdimen6\font
%	line: \printinunitsof{in}\prntlen{\textwidth} \printinunitsof{in}\prntlen{\linewidth}
% PGF images are by default rendered to seperate PDF files first, which
% are then included (faster & less memory problems); see makefile
% to include an image which can be referenced by \ref{fig:figure_label} use either:
%	\begin{figure}[ht]
%		\pgfimg{name_without_extension}
%		\caption{enter a caption here}
%		\label{fig:figure_label}
%	\end{figure}
% for PGF; for PNG use (possibly without [...]):
%		\includegraphics[width=\linewidth,keepaspectratio]{demo.png}

%% EQUATIONS
% small formulas can be placed between $ and $ , numbered equations
% are made like this:
%	\begin{equation} \begin{split}
%		\frac{\sum_{i=1}^{n-1}N_{i}N_{i+1}}{\sum_{i=1}^{n} N_{i}}
%		\label{eq:equation_label}
%	\end{split} \end{equation}
% where split let's you split the equation with \\ (no automatic way)

%% TABLES
% a simple table can be made like this:
%	\begin{table}
%		\begin{tabular}{| l | l | r | r |} \hline
%			underlined left cell & same & underlined right cell & same \\ \hline
%			left cell & left cell & right cell & right cell \\ \hline
%		\end{tabular}
%		\caption{enter a caption here}
%		\label{tab:table_label}
%	\end{table}

%% LISTS
%	\begin{enumerate} % or 'itemize'
%		\item
%		\item
%	\end{enumerate}


