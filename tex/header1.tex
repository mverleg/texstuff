%% LaTeX TOP general header for style and packages
% Mark Verleg, January 19th 2014

% switch from some USA size to A4
\documentclass[a4paper,twoside,UTF8]{article}

%% PACKAGES
% for missing packages, the lazy > 1GB way is: apt-get install texlive-full
% include packages
\usepackage{graphicx}	% for including images using \begin{figure}[ht] \includegraphics[keepaspectratio]{name.png}
\usepackage{amsmath}	% math formulas, with $ equation $ or \begin{equation}
\usepackage{multicol}	% to have selective areas, marged with \begin{multicols}{2}, have multiple columns
\usepackage{fancyhdr}	% allows adding headers and/or footers to the document
\usepackage{lastpage}	% get a reference to the last page (for page numbering)
\usepackage{adjustbox}	% scale things (pgf images) the same way as normal images
\usepackage{xcolor}	% provides \textcolor
\usepackage{layouts}	% just for outputting linewidths etc
\usepackage{ifthen}	% \setboolean for switching one/twoside
\usepackage{tikz}		% drawing library, related to pgf
\usepackage{pgf}		% to show .pgf images (as generated by matplotlib)
\usepackage{etoolbox}	% allows toggles and if statements
\usepackage[hidelinks]{hyperref}	% allows using simple URLs without infinite escape chars
\usepackage[backend=bibtex]{biblatex} % cite bibtex references
\usepackage{caption}	% align figure captions
\usepackage{braket}		% <a|b> notation, \Bra{}, \Braket{|}
\usepackage{url}		% get urls to display normally
\usepackage{fontspec}	% for setting fonts (for e.g. Chinese text)
% \usepackage{luaCJK}		% for foreign (CNY) character support
% \usepackage{fontspec}	% http://tex.stackexchange.com/questions/28642/frequently-loaded-packages-differences-between-pdflatex-and-lualatexx
% \usepackage{polyglossia}	% for non-English languages
% configure packages
\captionsetup[figure]{slc=off}	% align figure captions left
\newtoggle{electronic}
\newtoggle{finalversion}
\newtoggle{headers}
% \setCJKmainfont{AR PL UMing CN}	% on Ubuntu, use  fc-list :lang=zh  to find Chinese fonts

%% DEFINITIONS
% properties
\newcommand{\authorname}{Mark Verleg \textbackslash authorname}
\newcommand{\departmentname}{Department \textbackslash departmentname}
\newcommand{\departmentlogo}{}
\newcommand{\shorttitle}{Short title \textbackslash shorttitle}
\newcommand{\longtitle}{Long title \textbackslash longtitle}
%%\renewcommand{\abstract}{Abstract}
% functionality
\newcommand{\comment}[1]{}
\newcommand{\todo}[0]{\textcolor{red}{ TODO }}
\newcommand{\latin}[1]{{\it #1}}
\newcommand{\pgfimg}[1]{\includegraphics{#1.pdf}}
\newcommand{\topwidemarker}[0]{\par\noindent\rule{\dimexpr(0.5\textwidth-0.5\columnsep-0.4pt)}{0.4pt}\rule{0.4pt}{6pt}}
\newcommand{\bottomwidemarker}[0]{\vspace{\belowdisplayskip}\hfill\rule[-6pt]{0.4pt}{6.4pt}\rule{\dimexpr(0.5\textwidth-0.5\columnsep-1pt)}{0.4pt}}
\newcommand{\cny}[1]{{\setmainfont{WenQuanYi Zen Hei} #1 }} % find Chinese fonts on Ubuntu: fc-list :lang=zh

%% ALTERNATIVES
% toggles
\toggletrue{electronic}
\toggletrue{headers}
\toggletrue{finalversion}
% load images directly instead of through pdfs and makefile
%   \renewcommand{\pgfimg}[1]{\input{#1.pgf}}


