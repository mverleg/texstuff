
%% IMAGES
% images generated by matplotlib can be given the correct dimensions
% and fonts automatically by using MyMPL to generate them; to find text
% width and font settings, render in main body (with layouts package):
%	font: \fontname\font\ \the\fontdimen6\font
%	line: \printinunitsof{in}\prntlen{\textwidth} \printinunitsof{in}\prntlen{\linewidth}
% PGF images are by default rendered to seperate PDF files first, which
% are then included (faster & less memory problems); see makefile
% to include an image which can be referenced by \ref{fig:figure_label} use either:
%	\begin{figure}[!htbp]
%		\pgfimg{name_without_extension}
%		\caption{enter a caption here}
%		\label{fig:figure_label}
%	\end{figure}
% for PGF; for PNG use (possibly without [...]):
%		\includegraphics[width=\linewidth,keepaspectratio]{demo.png}

%% EQUATIONS
% small formulas can be placed between $ and $ , numbered equations
% are made like this:
%	\begin{equation} \begin{split}
%		\frac{\sum_{i=1}^{n-1}N_{i}N_{i+1}}{\sum_{i=1}^{n} N_{i}}
%		\label{eq:equation_label}
%	\end{split} \end{equation}
% where split let's you split the equation with \\ (no automatic way)

%% TABLES
% a simple table can be made like this:
%	\begin{table}
%		\begin{tabular}{| l | l | r | r |} \hline
%			underlined left cell & same & underlined right cell & same \\ \hline
%			left cell & left cell & right cell & right cell \\ \hline
%		\end{tabular}
%		\caption{enter a caption here}
%		\label{tab:table_label}
%	\end{table}

%% LISTS
%	\begin{enumerate} % or 'itemize'
%		\item
%		\item
%	\end{enumerate}

%% CODE
% include python statements or their result; everything is cached automatically; namespaces are parallel
% available commands:
% 	\py{2+2}	       runs code with all output (like interactive python interpreter)
% 	\pyc{print(2+2)}   runs code and shows stdout, like print statements; \begin{pycode}[namespace]
% 	\pyv{q = 42}       shows formatted python code; \begin{pyverbatim}[namespace]
% 	\pyb{q = 42}       shows formatted python code and runs it without output; \begin{pyblock}[namespace]
% 	\stdoutpythontex   shows output from the below
%	\begin{pyconsole}  emulates console python prompt (with >> and output)
%			\begin{pyblock}[namespace]
%				import sys
%				from bardeen.string import tex_escape
%				print(tex_escape(sys.version_info))
%			\end{pyblock}
%			\texttt{\printpythontex}

% pytex.context
% All non-console families import pythontex_utils.py, and create an instance of the PythonTEX utilities class called pytex. This provides various utilities for interfacing with LATEX and PythonTEX. The utilities class has an attribute context. This is a dictionary that can contain contextual information, such as page dimensions, from the TEX side. Values may also be accessed as attributes rather than as dictionary keys. To determine what contextual information is available, and for additional details, see \setpythontexcontext under Section 4.5. For working with contextual data, the utilities class provides pt_to_in(), pt_to_cm(), pt_to_mm(), and pt_to_bp() methods for converting from TEX points to other units.
% The context may only be set in the preamble
% \setpythontexcontext{textwidth=\the\textwidth, textheight=\the\textheight}


